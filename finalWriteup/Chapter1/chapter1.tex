%!TEX root = ../thesis.tex
%*******************************************************************************
%*********************************** First Chapter *****************************
%*******************************************************************************

\chapter{Introduction}  %Title of the First Chapter

\ifpdf
    \graphicspath{{Chapter1/Figs/Raster/}{Chapter1/Figs/PDF/}{Chapter1/Figs/}}
\else
    \graphicspath{{Chapter1/Figs/Vector/}{Chapter1/Figs/}}
\fi

In 2019, human civilisation was on the precipice of a natural disaster: \mbox{SARS-CoV-2} (\mbox{COVID-19}). First reported to the WHO on December 31st, it became officially recognised as a pandemic on March 11th 2020. As of the writing of this passage, 515 million cases and 6.24 million death have been recorded. This, however, is not the first time a pandemic has occurred, with the Black Death infamously killing a third of Europe's population and the Spanish Flu causing mass death throughout the world. Likewise, it is unlikely to be the last.

When such a disaster does strike, it is important to react quickly. Vaccinations are allowed accelerated timelines in development cutting development from years to month, and trials into potential treatments are encouraged with haste. Within the first stages of the pandemic, drugs such as hydroxychloroquine and bleach were amongst several that were promoted by the President of the United States of America demonstrating the desperation in finding therapeutic drugs against the virus.

In order to facilitate a more robust approach to finding treatments, the FDA instigated the Coronavirus Treatment Acceleration Program (CTAP) \cite{CTAP22}. Here, over 690 drugs are in the development stage with over 450 clinical trials underway to investigate the effectiveness, with 15 drugs currently authorised for emergency use and only one drug, remdesivir, with approval for use against \mbox{COVID-19} \cite{CTAP22}. Indeed, remdesivir is an important case. This drug was developed initially for hepatitis-c before being used for several other conditions until finally being used for \mbox{COVID-19} \cite{Joe20}. This demonstrates how a discovered drug can be repurposed for new diseases providing a cheap means of drug "redevelopment".

Investigations into pre-existing drugs, however, were slow and largely carried out through labour intensive mechanisms without a rational methodical testing regime. This added time to finding treatments to \mbox{COVID-19}. Time many did not have. A hopeful fulfilment of this problem is the "Robot Scientist"; a fully automated combination of software and hardware aimed at solving this problem. For the software side, a form of reinforcement machine learning is proposed: active learning. This is a methodology suited to fields with large amounts of unlabelled data which is difficult to label. In this case, the labelling requires chemical and biological experimentation costing both time and money. By using Active Learning, as few drugs as possible will be labelled within this stage to accurately predict the best drugs for the given problem. From here, clinical trials may begin.

