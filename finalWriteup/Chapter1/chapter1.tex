%!TEX root = ../thesis.tex
%*******************************************************************************
%*********************************** First Chapter *****************************
%*******************************************************************************

\chapter{Introduction}  %Title of the First Chapter

\ifpdf
    \graphicspath{{Chapter1/Figs/Raster/}{Chapter1/Figs/PDF/}{Chapter1/Figs/}}
\else
    \graphicspath{{Chapter1/Figs/Vector/}{Chapter1/Figs/}}
\fi

In 2019, human civilisation was on the precipice of a natural disaster: \mbox{SARS-CoV-2} (\mbox{COVID-19}). First reported to the World Health Organization (WHO) on December 31st, it became officially recognised as a pandemic on March 11th 2020. As of the writing of this passage, 515 million cases and 18 million excess deaths have been recorded \cite{Wan22,WHO22}. This, however, is not the first time a pandemic has occurred, with the Black Death infamously killing a third of Europe's population and the Spanish Flu causing mass death throughout the world. Likewise, it is unlikely to be the last.

When such a disaster does strike, it is important to react quickly. Vaccinations are developed and manufactured on accelerated timelines, cutting development time from years to months. Trials into potential treatments are encouraged with haste. When speed is not achieved with these measures, misinformation rapidly spreads. Within the first stages of the pandemic, drugs such as hydroxychloroquine and bleach were amongst several that were promoted by the President of the United States of America demonstrating the desperation in finding therapeutic drugs against the virus.

In order to facilitate a more robust approach to finding treatments, the FDA instigated the Coronavirus Treatment Acceleration Program \cite{CTAP22}. Here, over 690 drugs are in the development stage with over 450 clinical trials underway to investigate the effectiveness, with 15 drugs currently authorised for emergency use and only one drug, remdesivir, with approval for use against \mbox{COVID-19} \cite{CTAP22}. These results, with the timescale in which they were achieved, are suboptimal. This is due to the slow, laborious, methods used in investigations into pre-existing drugs. This resulted in delays in treatment. Time many did not have.

A potential solution to this problem is the "Robot Scientist" \cite{And10}; a fully automated combination of software and hardware dedicated to advancing science efficiently. For the software side, a form of reinforcement machine learning is proposed: batch active learning. Active learning attempts to determine a model for a problem by querying the fewest data points within a set of unlabelled data. This is a methodology suited to fields with large amounts of unlabelled data which is inherently difficult or expensive to label. In the case of drug discovery, the labelling requires chemical and biological experimentation costing both time and money. By using active learning, as few drugs as possible will be labelled within this stage to accurately predict the best drugs for the given problem. From here, accelerated, targetted clinical trials may begin.

Due to the large importance of time, many drugs may be tested in parallel. This becomes even more practical considering the existence of robotic testing facilities. This presents an additional problem: how does one set up a testing scheme for batches? Can the same techniques used in single-sample learning be transitioned across, or are more inventive methodologies required here?

Thus, the purpose of this thesis. To present an algorithm which may be used to discover effective drugs within a short period of time. Additionally, a framework will be developed that allows for different algorithms to be rigorously compared to each other for increased robustness. In doing so, several options are compared to each other, using a variety of techniques in order to produce a methodology more suited to quickly sampling large quantities of drugs against a target, such as the main protease of \mbox{COVID-19} (3-chymotrypsin-like protease) \cite{prot20}. Of the selection compared, RoDGer was found to be best, with the highest rate of initial learning. This method combines a technique called minimisation of the region of disagreement with clusterisation of the available samples and a greedy approach to aggressively find drugs with maximum activity.

To facilitate the combination of these different methods, as well as allowing robust comparison of different algorithms, a testing framework is introduced. This consists of two sections: a training section and a testing section. Using this, parameters may be introduced to allow for the optimal merging of the different methods before final being tested on a sample of datasets which were not used within the training process. This prevents data leakage: a technique commonly used within machine learning.

Using this framework, it was shown that popular algorithms for single sample active learning tend to perform worse than merely randomly sampling if employed on batches. Instead, a clustering algorithm was shown to be an extremely useful when scaling sample size. This allows commonly used methods to still be useful in batch learning. RoDGer clearly demonstrates this, combining both greedy methodology with regions of disagreement, two methods outperformed by random sampling, and provides a significant improvement on random sampling due to the combination of clustering.