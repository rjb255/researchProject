%!TEX root = ../thesis.tex
%*******************************************************************************
%****************************** Third Chapter **********************************
%*******************************************************************************
\chapter{Conclusion}

% **************************** Define Graphics Path **************************

\graphicspath{{Chapter6/Figs/Vector/}{Chapter6/Figs/}}




There are about 20000 drugs approved for pharmaceutical use, with more being trialled every year. However, there is hesitancy in using these drugs as treatments for emerging diseases. Instead, focus is given to designer drugs which may take years to develop. It is estimated 80\% of the cost in drug design and manufacture arises simply from the developing process. This was shown to be ineffective at the time of the SARS-CoV-2 outbreak whereby the use of existing drugs was slow, and often polluted with misinformation.

In order to combat these issues, the role of active learning cannot be understated. Reviewing drugs already in commercial production allows a fast role out upon approval. Secondly, research and development costs will reduce significantly due to low wastage. Thirdly, clinical trials could safely be accelerated by providing phase one trial data from the first trials the drug will have had to pass.

Thus, this thesis targetted the issue of an algorithm capable of executing batch active learning with batch sizes of 100. In doing so, several algorithms were tested and compared consisting of both parametric and non-parametric algorithms.