%!TEX root = ../thesis.tex
%*******************************************************************************
%****************************** Third Chapter **********************************
%*******************************************************************************
\chapter{Methodology}

% **************************** Define Graphics Path **************************

\graphicspath{{Chapter3/Figs/Vector/}{Chapter3/Figs/}}
\section{Outline}
The methodology presents a novel means of assessing different parametrised active learning methods on existing data sets, allowing for a robust answer into the use of active learning in drug rediscovery. Results can thus be given with a given belief. This approach has taken principles commonly used in machine learning and applied it to more traditional algorithmic methods.
\\
Firstly, a collection of pre-existing data sets, $X$, are used. $X$ is then split into two sub sets: $X_{\mathrm{train}}$ and $X_\mathrm{test}$. Similarly to machine learning, the former of these subsets is used in fitting the parameters of the equation, and the latter is used to provide a result without the risk of data leakage into the training set. This is represented in []. Parallelisation is used to efficiently train the algorithms allowing the time for training $\sim{}\mathcal{O}(c)$.
\\
In Section [], it was discussed that there are various methodologies of representing chemicals and drugs. ... (if time)
\section{Proof}
In order to demonstrate the effectiveness, a few data sets are used instead, and the program is executed function by function. To start with, the underlying custom functions will be demonstrated, followed by the algorithms and then finally the training framework.
\subsection{Custom Functions}
\blindtext[1]
\subsection{Active Learning Algorithms}
\blindtext[1]
\subsection{Training Framework}
\blindtext[1]
