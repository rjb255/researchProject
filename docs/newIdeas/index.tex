\documentclass[a4paper, english]{article}

\usepackage[style=alphabetic]{biblatex}
\usepackage{blindtext}
\usepackage[utf8]{inputenc}
\usepackage{babel,csquotes,xpatch}
\usepackage{amsmath}
\usepackage{algorithm2e}
\usepackage{booktabs}
\DeclareMathOperator*{\argmax}{argmax}
\DeclareMathOperator*{\argmin}{argmin}
\DeclareMathOperator*{\simm}{sim}

\addbibresource{bib/newIdeas.bib}
\begin{document}
\title{\Large{\textbf{Literature Review}}\\Batch Active Learning for Drug Discovery}
\author{rjb255}
\date{January 29, 2022}

\maketitle

\begin{abstract}
    \blindtext[1]{}
\end{abstract}



I propose~(\ref{eq:density}) where $N$ represents the dimensionality of the model space, and $r_i$ is the distance between $x_j$ and $x_i$. The next test point is given by () where $X_\mathrm{known}$ is the set of labelled data points and $X_\mathrm{unknown}$ is the set of available data points for testing.

\begin{equation}
    \label{eq:density}
    \rho_{x_j}=\sum_i{\frac{1}{r_{x_i,x_j}^N}}
\end{equation}

\begin{equation}
    x_\mathrm{next}=\argmin_{X_\mathrm{Unknown}}{\sum_{x=X_\mathrm{Known}}{\frac{1}{r_{X_\mathrm{Unknown},x}^N}}}
\end{equation}

\end{document}