\documentclass{article}
\begin{document}
\author{Ross Brown}
\title{Proposed Implementations/Changes}
\maketitle
\section{Evaluating}
\subsection{Current Issue}
The algorithms are running on 3 datasets at the moment for an arbitrary number of runs. They are then adjusted/modified manually to visually improve these results which is of little value.
\subsection{Solution}
Using a set of datasets (currently about 2000 datasets), $X$, portioned into training, validating, and testing subsets ($X_\mathrm{train}$, $X_\mathrm{test}$), more robust results can be produced. The idea for these subsets arises from the equivalent idea seen in machine learning.

$X_\mathrm{train}$ will be used to fit parameters in the algorithm and the $X_\mathrm{test}$ will have the fitted algorithm applied and the scoring reported. An $X_\mathrm{valid}$ is not deemed necessary at the moment.
\section{Scoring}
\subsection{Current Issues}
Simply taking the mse of the results does not fit in with the biological aspect: finding highly active compounds. The scoring thus needs to fulfil the following criteria:
\begin{itemize}
    \item Weighting to higher pXa
\end{itemize}
And would preferably meet the following:
\begin{itemize}
    \item Lightweight - would rather have computational power used on the active learning than the scoring.
    \item Independent of dataset size and distribution.
\end{itemize}
\subsection{Solutions}
\begin{itemize}
    \item Weighted mse:
          $$\sigma{}=\sum_i{w(y_i-\bar{y})^2}$$
          Where $w$ may be:
          $$w_i=y_i^\alpha$$
          It is unknown what $\alpha{}$ would be (so probably 1 according to Occam's razor).
    \item Number of `top' results to contain $a$ of the top $b$ true top values.
    \item Number of iterations needed to get either of the above below a predefined value.
\end{itemize}
Likely to choose the first solution as simple to implement with a target of 5 iterations. A validation could be used to determine the number of iterations (i.e. look at the rate of improvement and stop at a certain rate).
\end{document}